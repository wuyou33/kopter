\section{Theory of Helicopters}

\subsection{Equations of Motion for Rigid Airframe}

  As shown in \cite{Cooke}...

  The axes to be used are the helicopter body axes $(O,x,y,z)$ fixed in the helicopter and with its origin at the body axes centre. The components of velocity and force along the $Ox$, $Oy$ and $Oz$ axes are $U$, $V$, $W$ and $X$, $Y$, $Z$, respectively. The components of the rates of rotation about the same axes are $p$, $q$ and $r$ and the moments $L$, $M$ and $N$.

  Considering the position that the position of the centre of gravity $CG$ is given by the co-ordinates $d_x$, $d_y$ and $d_z$ relative to the body axes centre, the absolute velocity of $CG$ is given by $u'$, $v'$ and $w'$:

  \begin{equation}
  	u' = U - r d_y + q d_z \qquad v' = V - p d_z + r d_x \qquad w' = W - q d_x + p d_y
  \end{equation}

  \noindent
  and similarly, for the accelerations of the $CG$:

  \begin{equation}
  	a'_x = \dot{U}' - r v' + q w' \qquad a'_y = \dot{v}' - p w' + r u' \qquad a_z' = \dot{w}' - q u' + p v'
  \end{equation}

\subsection{Relationship between feathering law and flapping law for hovering}

  The flapping equation for hover is given by:$$M_{b,y_{A1}}^{a,E} + k_\beta \beta + I_\beta \left( \frac{\mathrm{d}^2\beta}{\mathrm{d}t^2} + \Omega^2\beta \right) + x_{\mathrm{GB}}M_P\Omega^2e\beta = 0, $$

  \noindent
  where the aerodynamic moment is given by:

  \begin{equation}
    M_{b,y_{A1}}^{a,E} = \rho a c \Omega^2 R^4 \left\{ \left[ -\frac{1}{6} + \frac{1}{4} \frac{e}{R} - \frac{1}{12} \left( \frac{e}{R} \right)^3 \right] \lambda_i +  \left[ \frac{1}{8} - \frac{1}{3}\frac{e}{R} ... \right] \right\},
  \end{equation}

  \noindent
  then, substituting $\psi = \Omega t$:

  \begin{equation}
    \frac{\mathrm{d}^2\beta}{\mathrm{d}\psi^2} + \eta_\beta \frac{\mathrm{d}\beta}{\mathrm{d}\psi} + \lambda_\beta^2\beta + \delta_\beta \lambda_i - \alpha_\beta \theta = 0,
  \end{equation}

  \noindent
  where $\eta_\beta$, $\delta_\beta$ and $\alpha_\beta$ are function of the Lock number $\gamma = \rho a c R^4 / I_\beta$, the blade eccentricity $e$ and the rotor radius $R$:

  \begin{eqnarray}
    \eta_\beta &=& \frac{\gamma}{8} \left[ 1 - \frac{8}{3}\frac{e}{R} + 2 \left(\frac{e}{R}\right)^2 - \frac{1}{3}\left(\frac{e}{R}\right)^4 \right], \\[6pt]
    \delta_\beta &=& \frac{\gamma}{8} \left[- \frac{4}{3} + 2 \frac{e}{R} - \frac{2}{3}\left(\frac{e}{R}\right)^3 \right], \\[6pt]
    \alpha_\beta &=& \frac{\gamma}{8} \left[1 - \frac{4}{3} \frac{e}{R} + \frac{1}{3}\left(\frac{e}{R}\right)^4 \right].
  \end{eqnarray}

  \noindent %damped natural frequency is written like w_d in the control systems Cranfield notes
  The blade natural damped natural frequency $\lambda_\beta$ is given by:$$\lambda_\beta = \sqrt{  1 + \frac{3}{2(1 - e/R)}\frac{e}{R} + 3 \frac{k_\beta}{R^3 m_P \Omega^2} \frac{1}{(1 - e/R)^3} }.$$ 
  
  \noindent
  The feathering control law is given by: $$ \theta(\phi) = \theta_0 + \theta_{1C}\cos{\phi} + \theta_{1S}\sin{\phi}$$

  Then, it can be assumed that the flapping angle can be reduced to its first harmonic: $$\beta(\phi) = \beta_0 + \beta_{1C}\cos{\phi} + \beta_{1S}\sin{\phi}.$$ Then the constant part $\beta_0$ is given by: $$\beta_0 = \frac{\alpha_\beta \theta_0 - \delta_\beta \lambda_{i0}}{\lambda_\beta^2},$$

  \noindent
  and $\beta_{1C}$ and $\beta_{1C}$ are given by:

  \begin{eqnarray}
    \beta_{1C} = \frac{\alpha_\beta}{(\lambda_\beta^2 - 1)^2 + \eta_\beta^2} [(\lambda_\beta^2 - 1)\theta_{1C} - \eta_\beta \theta_{1S}], \\[6pt]
    \beta_{1S} = \frac{\alpha_\beta}{(\lambda_\beta^2 - 1)^2 + \eta_\beta^2} [\eta_\beta \theta_{1C} + (\lambda_\beta^2 - 1)\theta_{1S}].
  \end{eqnarray}

